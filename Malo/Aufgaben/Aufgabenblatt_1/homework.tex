\documentclass{article}

\usepackage{fancyhdr}
\usepackage{extramarks}
\usepackage{amsmath}
\usepackage{amsthm}
\usepackage{ amssymb } % for \mathfrak{I}
\usepackage{mathtools} % for :=
\usepackage{stmaryrd}  % for doublebrackets
\usepackage{amsfonts}
\usepackage{tikz}
\usepackage[plain]{algorithm}
\usepackage{algpseudocode}

\usetikzlibrary{automata,positioning}

%
% Basic Document Settings
%

\topmargin=-0.45in
\evensidemargin=0in
\oddsidemargin=0in
\textwidth=6.5in
\textheight=9.0in
\headsep=0.25in

\linespread{1.1}

\pagestyle{fancy}
\chead{\hmwkClass\ (\hmwkTitle) - \hmwkGroup}
\rhead{\firstxmark}
\lfoot{\lastxmark}
\cfoot{\thepage}

\renewcommand\headrulewidth{0.4pt}
\renewcommand\footrulewidth{0.4pt}

\setlength\parindent{0pt}

%
% Create Problem Sections
%

\newcommand{\enterProblemHeader}[1]{
    \nobreak\extramarks{}{Aufgabe \arabic{#1} wird auf der nächsten Seite fortgesetzt\ldots}\nobreak{}
    \nobreak\extramarks{Aufgabe \arabic{#1} (fortsetzung)}{Aufgabe \arabic{#1} wird auf der nächsten Seite fortgesetzt\ldots}\nobreak{}
}

\newcommand{\exitProblemHeader}[1]{
    \nobreak\extramarks{Aufgabe \arabic{#1} (fortsetzung)}{Aufgabe \arabic{#1} wird auf der nächsten Seite fortgesetzt\ldots}\nobreak{}
    \stepcounter{#1}
    \nobreak\extramarks{Aufgabe \arabic{#1}}{}\nobreak{}
}

\setcounter{secnumdepth}{0}
\newcounter{partCounter}
\newcounter{homeworkProblemCounter}
\setcounter{homeworkProblemCounter}{1}
\nobreak\extramarks{Aufgabe \arabic{homeworkProblemCounter}}{}\nobreak{}

%
% Homework Problem Environment
%
% This environment takes an optional argument. When given, it will adjust the
% problem counter. This is useful for when the problems given for your
% assignment aren't sequential. See the last 3 problems of this template for an
% example.
%
\newenvironment{homeworkProblem}[1][-1]{
    \ifnum#1>0
        \setcounter{homeworkProblemCounter}{#1}
    \fi
    \section{Aufgabe \arabic{homeworkProblemCounter}}
    \setcounter{partCounter}{1}
    \enterProblemHeader{homeworkProblemCounter}
}{
    \exitProblemHeader{homeworkProblemCounter}
}

%
% Homework Details
%   - Title
%   - Due date
%   - Class
%   - Section/Time
%   - Instructor
%   - Author
%

\newcommand{\hmwkTitle}{Aufgabenbaltt 1}
\newcommand{\hmwkDueDate}{16.04.2019}
\newcommand{\hmwkClass}{Mathematische Logik}
\newcommand{\hmwkGroup}{Gruppe 12}
\newcommand{\hmwkAuthorName}{\textbf{Daniel Rupp(399831)} \and \textbf{Nicolas Heinen(399831)} \and \textbf{Third Name(399831)}}

%
% Title Page
%

\title{
    \vspace{2in}
    \textmd{\textbf{\hmwkClass:\ \hmwkTitle\ - \hmwkGroup}}\\
    \normalsize\vspace{0.1in}\small{\hmwkDueDate}\\
    \vspace{3in}
}

\author{\hmwkAuthorName}
\date{}

\renewcommand{\part}[1]{\textbf{\large Part \Alph{partCounter}}\stepcounter{partCounter}\\}

%
% Various Helper Commands
%

% Useful for algorithms
\newcommand{\alg}[1]{\textsc{\bfseries \footnotesize #1}}

% For derivatives
\newcommand{\deriv}[1]{\frac{\mathrm{d}}{\mathrm{d}x} (#1)}

% For partial derivatives
\newcommand{\pderiv}[2]{\frac{\partial}{\partial #1} (#2)}

% Integral dx
\newcommand{\dx}{\mathrm{d}x}

% Alias for the Solution section header
\newcommand{\solution}{\textbf{\large Lösung}}

% Probability commands: Expectation, Variance, Covariance, Bias
\newcommand{\E}{\mathrm{E}}
\newcommand{\Var}{\mathrm{Var}}
\newcommand{\Cov}{\mathrm{Cov}}
\newcommand{\Bias}{\mathrm{Bias}}
\newcommand{\interpret}[1]{\llbracket #1 \rrbracket^{\mathfrak{I}}}

\begin{document}

\maketitle

\pagebreak

\begin{homeworkProblem}[2]

\textbf{\Large a)}
\ \\

\textbf{Syntax}

\begin{tabbing}
	W \hspace{3em} \= Windows auf dem Computer\= \\
	M \> MacOS auf dem Computer \\
	L \> Linux auf dem Computer\\
	G \> Grafiktreiber installiert\\
	D \> Druckertreiber installiert\\
	K \> Konsole \\
	A \> Systemabsturz \\
	H \> Hausaufgaben gemacht\\ 
	S \> Computerspiele spielen \\
\end{tabbing}
\(
\tau = \lbrace W,M,L,G,D,K,A,H,S \rbrace
\)
\ \\

\textbf{Semantik}

\(
\forall x \in \tau, sei \ \mathfrak{I}: x \rightarrow \lbrace 0,1 \rbrace \ sodass \ \mathfrak{I}(x) = 1, wenn \ x \ zutrifft.
\)
\ \\
\ \\
\textbf{\Large b)}%----------------------------------------------------------
\ \\

\(
\Psi_{1} \coloneqq 	(W \wedge \neg M \wedge \neg L ) \vee 
					(\neg W \wedge  M \wedge \neg L ) \vee 
					(\neg W \wedge \neg M \wedge L ) \\
\Psi_{2} \coloneqq 	(\neg G \land S) \to A \\
\Psi_{3} \coloneqq 	H \to (D \land K) \\
\Psi_{4} \coloneqq 	(L \land G \land \neg D) \vee
					(L \land \neg G \land D) \vee
					\neg L \\
\Psi_{5} \coloneqq 	(W \land \neg K) \vee
					\neg W\\
\Psi_{6} \coloneqq 	(H \land S) \vee (\neg H \land \neg S)\\
\Psi_{7} \coloneqq 	\neg H \to A\\
\Psi_{8} \coloneqq 	\neg A \\
\)

\(
\llbracket \Psi_{1} \rrbracket^{\mathfrak{I}} \coloneqq
\max(	\min( \interpret{W}, \neg \interpret{M}, \neg \interpret{L}),
		\min( \neg \interpret{W}, \interpret{M}, \neg \interpret{L}),
		\min( \neg \interpret{W}, \neg \interpret{M}, \neg \interpret{L})
	) \\
\llbracket \Psi_{2} \rrbracket^{\mathfrak{I}} \coloneqq
\max(	\neg \min(\interpret{\neg G}, \interpret{S}),
		\interpret{A} 
	)\\
\llbracket \Psi_{3} \rrbracket^{\mathfrak{I}} \coloneqq
\max(	\interpret{A} ,
	\neg \min(\interpret{D}, \interpret{K})
	)\\
\llbracket \Psi_{4} \rrbracket^{\mathfrak{I}} \coloneqq
\max(	\min( \interpret{L} , \interpret{G} , \interpret{\neg D}),
		\min( \interpret{L} , \interpret{\neg G} , \interpret{D}),
		\interpret{\neg L} 
	)\\
\llbracket \Psi_{5} \rrbracket^{\mathfrak{I}} \coloneqq
\max(	\min(\interpret{W},\interpret{\neg K}),
		\interpret{\neg W}  
	)\\
\llbracket \Psi_{6} \rrbracket^{\mathfrak{I}} \coloneqq
\max(	\min(\interpret{H}, \interpret{S}  ),
		\min(\interpret{\neg H} , \interpret{\neg S})
)\\
\llbracket \Psi_{7} \rrbracket^{\mathfrak{I}} \coloneqq
\max (\interpret{H}, \interpret{A}) \\
\llbracket \Psi_{8} \rrbracket^{\mathfrak{I}} \coloneqq
\interpret{\neg A} \\
\)

\textbf{\Large c)}%----------------------------------------------------------
\ \\

    \begin{enumerate}
        \item Aus \(\interpret{\Psi_{8}}\) folgt, dass \(\interpret{A}\) = 0 gilt.
        \item \(\interpret{A}\) = 0 Eingesetzt in \(\interpret{\Psi_{2}}\) 																		und \(\interpret{\Psi_{7}}\) 
    	\begin{enumerate}
    		\item \(\interpret{\Psi_{2}} \coloneqq \neg \min \interpret{\neg G}, \interpret{S})\)
    		\item \(\interpret{\Psi_{7}} \coloneqq \interpret{H}\)
    	\end{enumerate}
    \end{enumerate}

\end{homeworkProblem}

\newpage

\end{document}
